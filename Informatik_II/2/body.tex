%% Einleitung

\begin{tabularx}{\textwidth}{Xr}
  Constantin Lazari, Marco Wettstein & \today\\
\end{tabularx}

%% Fragen
\begin{questions}
  \question
	Geben Sie (graphisch) einen DEA mit Alphabet $\Sigma = \{0, 1, 2\}$ an, der genau
	die W�rter aus $\Sigma^*$ akzeptiert welche geraden nat�rlichen Zahlen in der Tern�rdarstellung (Basis 3 Darstellung) entsprechen.

	Beispiel: Der Automat akzeptiert beispielsweise $1101221$, verwirft aber $1010010$

	\begin{solutionordottedlines}[2cm]
	$0 \mbox{ lesen: } \cdot 3$\\
	$1 \mbox{ lesen: } \cdot 3 + 1$\\
	$2 \mbox{ lesen: } \cdot 3 + 2$\\
		\begin{center}
			\begin{tikzpicture}[->,>=stealth',shorten >=1pt,auto,node distance=5cm, semithick]
			\tikzstyle{every state}=[fill=black!10, align=center, minimum width=3cm]
			
			\node[state]	(A)						{$q_0$\\(gerade)};
			\node[state]	(B) [right of=A]		{$q_1$\\(ungerade)};

			\path 	
					(A) edge [loop above] 	node {$0$} (A)
						edge [bend left]	node {$1$} (B)
						edge [loop below]	node {$2$} (A)
 					(B) edge [loop above]	node {$0$} (B)
 						edge [bend left]	node {$1$} (A)
 						edge [loop below]	node {$2$} (B);
			\end{tikzpicture}
		\end{center}
	\end{solutionordottedlines}
	\pagebreak

   %%%%%%%%%%%%%%%%%%%%%%%%%%%%%%%%%%%%%%%%%%%%%%%%%%%%
	\question
	Es sei $\Sigma$ ein beliebiges Alphabet und $A \subset \Sigma^*$ eine Sprache.
	Ist folgende Aussage wahr? Begr�nden Sie Ihre Antwort.
	
	\begin{equation}
		A \mbox{ ist regul�r } \Leftrightarrow (\Sigma^*\setminus A) \mbox{ ist regul�r}
	\end{equation}

	\begin{solutionordottedlines}[2cm]
	$\Sigma^* \setminus A$ ist das Inverse von A (und andersrum).

	1. \enquote{$\Rightarrow$}\\
	Es l�sst sich ein deterministischer, endlicher Automat (DEA a) bauen, der pr�ft,
	ob ein gegebenes Wort bzw. ein regul�rer Ausdruck von Element von $A$ ist.

	Es l�sst sich auch ein Automat (DEA b) bauen, der:
	\begin{enumerate}
		\item alle W�rter von $\Sigma^*$ akzeptiert
		\item alle akzeptierten W�rter DEA a �bergibt
		\item Falls:
		\begin{enumerate}
			\item das Wort von DEA a akzeptiert wird, es f�r ung�ltig erkl�rt
			\item ansonsten das Wort f�r g�ltig erkl�rt.
		\end{enumerate}
	\end{enumerate}
	Somit ist die Implikation nach rechts bewiesen.

	2. \enquote{$\Leftarrow$}\\
	Sofern $\Sigma^*$ regul�r ist, l�sst sich ein deterministischer, endlicher Automat bauen,
	der pr�ft, ob ein Wort Element von $\Sigma^*$ ist und falls nicht in den 
	Zustand \enquote{$w \notin \Sigma^*$} �bergeht.

	Falls, die $\Sigma^*$ Pr�fung erfolgreich verl�uft, kann im n�chsten Schritt
	gepr�ft werden, ob das Wort $\in A$ ist. (Falls ja, Zustand \enquote{$w \in A$},
	sonst Zustand \enquote{$w \notin A$}).

	Somit kann der Automat entscheiden, ob ein Wort ein Element von $\Sigma^* \setminus A$ ist.
	Damit ist auch die Implikation nach links bewiesen.

	Im Ergebnis ist die Aussage damit richtig, sofern $\Sigma^*$ auch regul�r ist.
	\end{solutionordottedlines}
	\pagebreak

   %%%%%%%%%%%%%%%%%%%%%%%%%%%%%%%%%%%%%%%%%%%%%%%%%%%%
	\question
	Die Zustands�bergangfunktion $\delta$ vom NEA $A = (\{q_0, q_1, q_2\}, \{0, 1\}, \delta, \{q_0\}, \{q_0\})$
	ist durch folgende Tabelle gegeben:

	\begin{center}
		\begin{tabular}{c|c|c}
			$\delta$ 	& $0$ 			& $1$\\\hline
			$q_0$ 		& $\{q_0\}$ 	& $\{q_1, q_2\}$\\\hline
			$q_1$		& $\{q_0\}$		& -- \\\hline
			$q_2$		& --			& $\{q_0\}$
		\end{tabular}
	\end{center}

	\begin{parts}
		\part Zeichnen Sie das Zustands�bergangsdiagramm von $A$.
		\begin{solutionordottedlines}[2cm]
		\begin{center}
			\begin{tikzpicture}[->,>=stealth',shorten >=1pt,auto,node distance=3cm, semithick]
			\tikzstyle{every state}=[fill=black!10, align=center]

			\node[state] 	(A)						{$q_0$};
			\node[state]	(B) [below left of=A]	{$q_1$};
			\node[state]	(C) [below right of=A]	{$q_2$};

			\path 	
					(A) edge [loop above] 	node {$0$} (A)
						edge [bend left]  	node {$1$} (B)
						edge [bend left]	node {$1$} (C)
					(B) edge [bend left]	node {$0$} (A)
					(C)	edge [bend left]	node {$1$} (A);
			\end{tikzpicture}
		\end{center}
		
		\end{solutionordottedlines}
		
		\part Beschreiben Sie die vom Automaten akzeptierte Sprache $L(A)$.
		\begin{solutionordottedlines}[2cm]
		Als regul�rer Ausdruck: $L(A) = (0 + (10) + (11))^*$
		\end{solutionordottedlines}

		\part Konstruieren Sie den zu $A$ �quivalenten DEA $A_D$. Verwenden Sie dazu
		die Teilmengenkonstruktion (siehe Hopcroft et al. S. 70ff. -- Kopie der S. auf Moodle).
		\begin{solutionordottedlines}[2cm]
		Teilmengenkonstruktion:
		\begin{center}
			\begin{tabular}{c|c|c|c}
					& $\delta$ 		& ~~~$0$~~~ & ~~~$1$~~~ \\\hline
			A		& $\emptyset$ 	& $\emptyset$ & $\emptyset$ \\\hline
			B		& $\{q_0\}$		& B		& G \\\hline
			C		& $\{q_1\}$		& -- 	& -- \\\hline
			D		& $\{q_2\}$		& --	& -- \\\hline
			E		& $\{q_0, q_1\}$ & --	& -- \\\hline
			F		& $\{q_0, q_2\}$ & -- 	& -- \\\hline
			G		& $\{q_1, q_2\}$ & B	& B \\\hline
			H		& $\{q_0, q_1, q_2\}$ & -- & -- \\
			\end{tabular}
		\end{center}

		Darstellung ($q_1$ von $A \neq q_1$ von $A_D$):
		\begin{center}
			\begin{tikzpicture}[->,>=stealth',shorten >=1pt,auto,node distance=4cm, semithick]
			\tikzstyle{every state}=[fill=black!10]

			\node[state] 	(A)						{$q_0$};
			\node[state]	(B) [right of=A]		{$q_1$};

			\path 	
					(A) edge [loop left] 	node {$0$} (A)
						edge [bend left]  	node {$1$} (B)
					(B) edge [bend left]	node {$0, 1$} (A);
			\end{tikzpicture}
		\end{center}
		\end{solutionordottedlines}


	\end{parts}
\end{questions}

