%% Einleitung

\begin{tabularx}{\textwidth}{Xr}
  Constantin Lazari, Marco Wettstein & \today\\
\end{tabularx}

%% Fragen
\begin{questions}
  \question
  Ziel dieser Aufgabe ist, dass Sie sich untereinander austauschen, im Buch und Ihren Unterlagen nachschlagen und sich gedanklich mit der Definition eines Algorithmus auseinandersetzen.

	\begin{parts}
		\part Spontan und ohne Hilfsmittel: Schreiben Sie alleine auf, wie sie den Algorithmus-Begriff definieren w�rden.
		\begin{solutionordottedlines}[2cm]
		Ein Algorithmus ist eine vollst�ndige Reihe von Arbeitsanweisungen, die aus einem definierten Input einen definierten Output erzeugt.

		\end{solutionordottedlines}

		\part �berlegen Sie sich welche Eigenschaften eienn Algorithmus ausmachen.
		\begin{solutionordottedlines}[2cm]
		Die Arbeitsansweisungen eines Algorithmuses m�ssen dabei folgende Kriterien erf�llen:
		\begin{enumerate}
		 \item Vollst�ndigkeit: Die Anweisungen m�ssen jeden auszuf�hrenden Schritt enthalten.
		 \item Unmissverst�ndlichkeit: Jede Arbeitsanweisung muss eindeutig interpretierbar sein.
		 \item Machbarkeit: Jede Arbeitsanweisung muss ausf�hrbar sein. Unm�gliche Anweisungen sind unzul�ssig.
		\end{enumerate}
		\end{solutionordottedlines}

		\part Diskutieren Sie Ihre Definition in der Gruppe.
		\begin{solutionordottedlines}[2cm]
		Gefallene Stichworte:
		\begin{enumerate}
		 \item Regelwerk mit Ein- und Ausgabe
		 \item Kochrezept
		 \item vollst�ndige Anweisungen
		 \item Verfahren zur Probleml�sung
		 \item Beschreibung von Funktionen
		 \item Befehlssequenz
		\end{enumerate}

		\end{solutionordottedlines}

		\part �berarbeiten Sie Ihre erste Version des Begriffs Algorithmus und erstellen Sie ihre eigene Definition.
		\begin{solutionordottedlines}[2cm]
		 Ein Algorithmus ist eine endliche formale Beschreibung von Funktionen oder Operationen, die aus elementaren Anweisungen oder Konstrukten besteht.
		\end{solutionordottedlines}

		\part Vergleichen Sie ihre Definition mit den Unterlagen vom Theorieunterricht
		\begin{solutionordottedlines}[2cm]
		Erg�nzende Eigenschaften:
		\begin{itemize}
		 \item Vollst�ndigkeit: fraglich, ob Vollst�ndigkeit nicht eher eine Relation zwischen Spezifikation und Algorithmus ist
		 \item Machbarkeit: diskutabel. Es gibt Algorithmen f�r Quantencomputer, die in dieser Form (noch) nicht existieren.
		\end{itemize}

		\end{solutionordottedlines}
	\end{parts}

   %%%%%%%%%%%%%%%%%%%%%%%%%%%%%%%%%%%%%%%%%%%%%%%%%%%%
	\question
	Entwerfen Sie einen Algorithmus zur Berechnung der Fibonacci-Zahlen
	\begin{parts}
		\part Stellen Sie den Algorithmus graphisch dar
		\begin{solutionordottedlines}[2cm]
		
\nassiwidth =\columnwidth \setiftext {J}{N}
\STRUCT { Fibonacci Sequenz }{ Mathematischer Algorithmus }{
  \ACTION{Einlesen von $n$ (die $n$-te Fibonacci-Zahl)}
  \IF{$n > 2$}
    \THEN{\ACTION{Gebe zur�ck: Fibonacci Sequenz mit $n - 1$}}
    \ELSE{\ACTION{Gebe zur�ck: 1}}
  \ENDIF
}


		\end{solutionordottedlines}
		
		\part Geben Sie eine text-basierte Darstellung Ihres Algorithmus an.
		\begin{solutionordottedlines}[2cm]
		Algorithmus: Fibonacci Sequenz ($n$):
		\begin{enumerate}
		 \item Falls $n > 2$: Gebe Fibonacci Sequenz $n-1$
		 \item Sonst: Gebe 1 zur�ck.
		\end{enumerate}

		\end{solutionordottedlines}

		\part Diskutieren Sie die Vor- und Nachteile der verschiedenen Darstellungsformen.
		\begin{solutionordottedlines}[2cm]
		
		\end{solutionordottedlines}
	\end{parts}

   %%%%%%%%%%%%%%%%%%%%%%%%%%%%%%%%%%%%%%%%%%%%%%%%%%%%
	\question Datenobjekte
	\begin{parts}
		\part Welche Datenobjekte haben Sie f�r Ihren Algorithmus eingesetzt?
		\begin{solutionordottedlines}[2cm]
		\end{solutionordottedlines}
		
		\part Handelt es sich um elementare oder strukturierte?
		\begin{solutionordottedlines}[2cm]
		\end{solutionordottedlines}

		\part Konstanten oder Variablen?
		\begin{solutionordottedlines}[2cm]
		\end{solutionordottedlines}
	
		\part Diskutieren Sie ihre Gedanken in der Gruppe.
		\begin{solutionordottedlines}[2cm]
		\end{solutionordottedlines}
      \end{parts}
      
   %%%%%%%%%%%%%%%%%%%%%%%%%%%%%%%%%%%%%%%%%%%%%%%%%%%%
	\question
	Vom Algorithmus zum Programm
	\begin{parts}
		\part Implementieren Sie den Algorithmus. Jedoch nicht ihren eigenen, sondern den einer anderen Gruppe.
		Nutzen Sie hierzu die erarbeite Darstellung des Algorithmus der anderen Gruppe.
		\begin{solutionordottedlines}[2cm]
		\end{solutionordottedlines}
		
		\part Falls Ihr Programm nicht oder nicht korrekt l�uft, dann diskutieren Sie mit der Gruppe von welcher Sie den Algorithmus erhalten haben,
		was dier Ursache der Fehler sein k�nnte.
		\begin{solutionordottedlines}[2cm]
		\end{solutionordottedlines}

	\end{parts}
	
\end{questions}

