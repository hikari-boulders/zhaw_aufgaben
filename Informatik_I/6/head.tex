\usepackage[utf8]{inputenc}
\usepackage[pdftex]{graphicx} 
\usepackage{microtype}
\usepackage[pdfborder={0 0 0}, plainpages=false, pdfpagelabels]{hyperref} 
\usepackage[ngerman]{babel}
\usepackage[babel]{csquotes}
\usepackage{tabularx} 
\usepackage{amsmath} 
\usepackage{eurosym}
\usepackage{tikz}
\usetikzlibrary{arrows,automata}
\usepackage{lscape, pdflscape}
\usepackage{booktabs}


\usepackage{lmodern} %Type1-Schriftart fuer nicht-englische Texte
\hyphenation{eine einer eines} % Trennung von eine, einer, eines vermeiden

\usepackage{color}
\usepackage{stmaryrd}


%% Listings %%%%%%%%%%%%%%%%%%%%%%%%%%%%%%%%%%%%%%%%%%%%%%%%%
%\usepackage{verbatim}
\usepackage{listings}
\lstloadlanguages{[LaTeX]TeX}
{\lstset{%
  basicstyle=\footnotesize\ttfamily,
  commentstyle=\slshape\color{green!50!black},
  keywordstyle=\bfseries\color{blue!50!black},
  identifierstyle=\color{blue},
  stringstyle=\color{orange},
  escapechar=\#,
  emphstyle=\color{red}}
}

{
  \lstset{%
    basicstyle=\ttfamily,
    keywordstyle=\bfseries,
    commentstyle=\itshape,
    escapechar=\#,
    emphstyle=\bfseries\color{red}
  }
}

\hypersetup{
	pdfauthor   = {Constantin Lazari, Marco Wettstein (BA Informatik 2012, Zürich)},
	pdftitle    = {Informatik I: Übung 6},
	pdfsubject  = {Informatik Zahlen},
	pdfkeywords = {Codierung, Komprimierung},
	pdfcreator  = {Kile},
	pdfproducer = {pdflatex},
	colorlinks 	= false
} 

\setlength{\parindent}{0em}
\setlength{\parskip}{0.75em}

%% Exam Settings
\pagestyle{headandfoot}
%\firstpageheader{Benutzer/innen im Umgang mit Informatikmitteln instruieren}{}{Lernkontrolle 1}
\firstpageheader{Informatik I (2012)}{}{Übung 6}
\firstpageheadrule

%\runningheader{Benutzer/innen im Umgang mit Informatikmitteln instruieren}{}{Lernkontrolle 1}
\runningheader{Informatik I}{}{
\ifprintanswers
  Lösung Übung 6
\else
  Übung 6
\fi
}
\runningheadrule

\firstpagefooter{}{Seite \thepage\ von \numpages}{}
\firstpagefootrule

\runningfooter{}{Seite \thepage\ von \numpages}{}
\runningfootrule

\pointsinrightmargin
\pointpoints{Punkt}{Punkte}
\bonuspointpoints{Bonuspunkt}{Bonuspunkte}
\renewcommand{\solutiontitle}{\noindent\textbf{Lösung:}\par\noindent}

\CorrectChoiceEmphasis{\bfseries}
\renewcommand\choicelabel{$\boxempty$}
