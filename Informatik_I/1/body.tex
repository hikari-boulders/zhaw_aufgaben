%% Einleitung

\begin{tabularx}{\textwidth}{Xr}
  Constantin Lazari, Marco Wettstein & 26. September 2012\\
\end{tabularx}

%% Fragen
\begin{questions}
  \question
  Was ist der grunds�tzliche Unterschied einer von Neumann- und Harvard-Architektur?
  
  \begin{solutionordottedlines}[2cm]
Ein Harvard-Architektur trennt den Arbeits- bzw. Datenspeicher physisch vom Programmspeicher.
Bei der klassischen und h�ufigeren von Neumann-Architektur gibt es nur einen Speicher.
  \end{solutionordottedlines}
 
   %%%%%%%%%%%%%%%%%%%%%%%%%%%%%%%%%%%%%%%%%%%%%%%%%%%%
	\question
	Welche Vorteile bietet eine Programmierung in Assembler gegen�ber einer in Maschinensprache?

  \begin{solutionordottedlines}[2cm]
Die Maschinensprache besteht eigentlich nur aus 0 und 1 bzw. und letztendlich aus Spannungsunterschieden. 
Das ist an sich nicht lesbar.

Assembler hingegen verwendet (sehr) kurze Buchstabenkombinationen, die f�r bestimmte Anweisungen in Maschinensprache stehen. 
So gesehen ist Assembler eine Metasprache, die f�r (ge�bte) Menschen lesbar ist.
  \end{solutionordottedlines}

\question
Was ist der Unterschied zwischen einer Rechenmaschine und einem Computer?

  \begin{solutionordottedlines}[2cm]
Eine Rechenmaschine arbeitet mechanisch (in der Regel mit Hilfe von Zahnr�dern). 
Ein Computer hingegen arbeitet elektronisch (fr�her mit Hilfe von R�hren, sp�ter Transistoren und jetzt mit integrierten Schaltkreisen).
  \end{solutionordottedlines}

\titledquestion{Mooresche Gesetz}
  \thequestiontitle
  \begin{parts}
  \part
  Wo finden Sie das \enquote{Mooresche Gesetz} in der heutigen Zeit?
  \begin{solutionordottedlines}[2cm]
Das \enquote{Mooresche Gesetz} �ussert sich h�ufig dadurch, 
dass sich entweder die Kosten von technischen Ger�ten mit der Zeit halbiert 
oder die Ger�te leistungsf�higer (=komplexer im Aufbau) werden.
  \end{solutionordottedlines} 

  \part
  Geben Sie ein Beispiel an.
  \begin{solutionordottedlines}[2cm]
Ein gutes Beispiel sind Flash-Speicher wie USB-Sticks oder SD-Speicher 
(z.\,B. f�r digitale Video- oder Fotokameras), welche regelm�ssig g�nstiger werden. 
Gleichzeitig kommen neue Speichergr�ssen hinzu; dabei ist der Preis f�r das jeweils gr�sste Modell relativ konstant -- die Speichergr�sse verdoppelt sich jedoch jeweils.
  \end{solutionordottedlines}

  \end{parts}

\end{questions}

