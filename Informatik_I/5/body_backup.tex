%% Einleitung

\begin{tabularx}{\textwidth}{Xr}
  Constantin Lazari, Marco Wettstein & \today\\
\end{tabularx}

%% Fragen
\begin{questions}
	\titledquestion{Das Wort $1010 001$ mit dem korrekt gesetzten \textit{Paritäts-Bit 1} (für
die gerade einfache Parität) wird bei der Übertragung durch Fehler verfälscht. Geben Sie an, ob der oder die Fehler erkannt
werden und begründen Sie Ihre Antwort kurz (Stichwort).
}
	\thequestiontitle
	\begin{parts}
		\part $1010\,0011 \Rightarrow 1010\,0\underline{1}11 \Rightarrow$
		\begin{solutionordottedlines}[2cm]
			Es handelt sich um einen 1-Bit-Fehler. Dieser kann nur erkannt werden, 
			wenn die Hamming-Distanz zwischen den gültigen Code-Wörtern grösser als 1 ist.
			Da in der Aufgabenstellung darüber keine Aussage gemacht wird, muss die Antwort soweit offen bleiben.
		\end{solutionordottedlines}
		
		\part $1010\,0011 \Rightarrow \underline{0}010\,\underline{1}011 \Rightarrow $
		\begin{solutionordottedlines}[2cm]
			Es handelt sich um einen 2-Bit-Fehler. Dieser kann nur erkannt werden, 
			wenn die Hamming-Distanz zwischen den gültigen Code-Wörtern grösser als 2 ist.
			Ausserdem muss der Fehler zu einem ungültigen Code-Wort führen.
			Da in der Aufgabenstellung darüber keine Aussage gemacht wird, muss die Antwort soweit offen bleiben.
		\end{solutionordottedlines}

		\part $1010\,0011 \Rightarrow 10\underline{0}0\,\underline{1}011 \Rightarrow $
		\begin{solutionordottedlines}[2cm]
			Es handelt sich um einen 2-Bit-Fehler. Dieser kann nur erkannt werden, 
			wenn die Hamming-Distanz zwischen den gültigen Code-Wörtern grösser als 2 ist.
			Ausserdem muss der Fehler zu einem ungültigen Code-Wort führen.
			Da in der Aufgabenstellung darüber keine Aussage gemacht wird, muss die Antwort soweit offen bleiben.
		\end{solutionordottedlines}
		
	\end{parts}

	%%%%%%%%%%%%%%%%%%%%%%%%%%%%%%%%%%%%%%%%%%%%%%%%%%%%
	\pagebreak
	\question Die Daten werden mit einer 2-Dimensionalen Parität (gerade) gesichert:
	\begin{itemize}
		\item das 1. Paritäts-Bit entspricht der einfachen Parität für jeweils einem 7-Bit-Wort
		\item das 2. Paritäts-Bit ist das jeweils i-te Bit (Spalte)
	\end{itemize}
	Gegeben sind drei übertragene Blöcke mit Übertragungsfehlern.
	Werden diese erkannt? Begründen Sie Ihre Antwort.

	\begin{parts}
		\part
		\begin{tabular}{|c|c|c|c|c|c|c|c|}
		\hline
			1 & 1 & 0 & 0 & 0 & 0 & 1 & \textbf{1} \\\hline %1
			1 & 1 & 0 & 0 & 0 & 1 & 0 & \textbf{1} \\\hline %2
			1 & 1 & 0 & 1 & 0 & 1 & 1 & \textbf{0} \\\hline %3
			1 & 1 & 0 & 0 & 1 & 0 & 0 & \textbf{1} \\\hline %4
			1 & 1 & 0 & 0 & 1 & 0 & 1 & \textbf{0} \\\hline %5
			1 & 1 & 0 & 1 & 1 & 1 & 0 & \textbf{0} \\\hline %6
			1 & 1 & 0 & 0 & 1 & 1 & 1 & \textbf{1} \\\hline %7
			1 & 1 & 0 & 1 & 0 & 0 & 0 & \textbf{1} \\\hline %8
			\textbf{0} & \textbf{0} & \textbf{0} & \textbf{1} & \textbf{0} & \textbf{0} & \textbf{0} & \textbf{1}\\\hline
		\end{tabular}

		\begin{solutionordottedlines}[2cm]
			Erkannt werden kann folgendes:
			\begin{enumerate}
				\item Zeile 3 enthält fünf Einsen, Paritätsbit müsste 1 sein.
				\item Zeile 6 enthält fünf Einsen, Paritätsbit müsste 1 sein. 
			\end{enumerate}
			Offenbar enthalten Zeilen 3 und 6 Fehler $\Rightarrow$ Nicht-korrigierbarer Mehrbitfehler.
		\end{solutionordottedlines}

		\part
		\begin{tabular}{|c|c|c|c|c|c|c|c|}
		\hline
			1 & 1 & 0 & 0 & 0 & 0 & 1 & \textbf{1} \\\hline %1
			1 & 1 & 0 & 0 & 0 & 1 & 0 & \textbf{1} \\\hline %2
			1 & 1 & 0 & 1 & 0 & 1 & 1 & \textbf{0} \\\hline %3
			1 & 1 & 0 & 0 & 1 & 0 & 0 & \textbf{1} \\\hline %4
			1 & 1 & 0 & 0 & 1 & 1 & 1 & \textbf{0} \\\hline %5
			1 & 1 & 0 & 0 & 1 & 1 & 0 & \textbf{0} \\\hline %6
			1 & 1 & 1 & 0 & 1 & 1 & 1 & \textbf{1} \\\hline %7
			1 & 1 & 0 & 1 & 0 & 0 & 0 & \textbf{1} \\\hline %8
			\textbf{0} & \textbf{0} & \textbf{0} & \textbf{1} & \textbf{0} & \textbf{0} & \textbf{0} & \textbf{1}\\\hline
		\end{tabular}

		\begin{solutionordottedlines}[2cm]
		  Erkannt werden kann folgendes:
		  \begin{enumerate}
		   \item Zeile 3 enthält fünf Einsen, Paritätsbit müsste 1 sein.
		   \item Zeile 5 enthält fünf Einsen, Paritätsbit müsste 1 sein.
		   \item Zeile 7 enthält sechs Einsen, Paritätsbit müsste 0 sein.
		   \item Spalte 3 enthält eine Eins, Paritätsbit müsste 1 sein.
		   \item Spalte 4 enthält 2 Einsen, Paritätsbit müsste 0 sein.
		   \item Spalte 6 enthält 5 Einsen, Paritätsbit müsste 1 sein.
		  \end{enumerate}
		  Es handelt sich mindestens um einen nicht-korrigierbaren 3-Bit-Fehler.
		\end{solutionordottedlines}

		\part
		\begin{tabular}{|c|c|c|c|c|c|c|c|}
		\hline
			1 & 1 & 0 & 0 & 0 & 0 & 1 & \textbf{1} \\\hline %1
			1 & 1 & 0 & 0 & 0 & 1 & 0 & \textbf{1} \\\hline %2
			1 & 1 & 0 & 0 & 0 & 1 & 1 & \textbf{0} \\\hline %3
			1 & 1 & 0 & 0 & 1 & 0 & 0 & \textbf{1} \\\hline %4
			1 & 1 & 0 & 0 & 1 & 0 & 1 & \textbf{0} \\\hline %5
			1 & 1 & 0 & 0 & 1 & 1 & 0 & \textbf{0} \\\hline %6
			1 & 1 & 0 & 0 & 1 & 1 & 1 & \textbf{1} \\\hline %7
			1 & 1 & 0 & 1 & 0 & 0 & 0 & \textbf{1} \\\hline %8
			\textbf{0} & \textbf{0} & \textbf{0} & \textbf{1} & \textbf{0} & \textbf{0} & \textbf{0} & \textbf{1}\\\hline
		\end{tabular}

		\begin{solutionordottedlines}[2cm]
		  In diesem Fall sind keine Fehler erkennbar, weil die Paritätsbits alle stimmen.
		  Daraus kann leider nicht gefolgert werden, dass keine Fehler da sind.
		  Beispielsweise bei einer Anordnung im Viereck, könnte der Block mindestens 4 Einzelbitfehler aufweisen.
		\end{solutionordottedlines}

		
	\end{parts}

	%%%%%%%%%%%%%%%%%%%%%%%%%%%%%%%%%%%%%%%%%%%%%%%%%%%%
	\question Gegeben ist:
	\begin{itemize}
		\item das Generatorpolynom $g = x^5 + x^2 + x + 1$
		\item eine zu übertragende Bitfolge: $1101\,0111\,0111$
	\end{itemize}


	\begin{parts}
		\part Geben Sie $I(x)$ an
		\begin{solutionordottedlines}[2cm]
		\begin{equation*}
		 I(x) = x^{11} + x^{10} + x^8 + x^6 + x^5 + x^4 + x^2 + x + 1
		\end{equation*}

		\end{solutionordottedlines}
		\begin{landscape}
		\part Berechnen Sie $M(x)$ sowie $R(x)$ mit dem CRC-Verfahren
		\begin{solutionordottedlines}[2cm]
		  \begin{align*}
		   M(x)& = (x^{11} + x^{10} + x^8 + x^6 + x^5 + x^4 + x^2 + x + 1) \cdot x^5\\
		   & = x^{16} + x^{15} + x^{12} + x^{10} + x^9 + x^7 + x^6 + x^5
		   \end{align*}
			Der Rest ergibt sich aus der Polynom-Division von:
				$R(x) = x^{16} + x^{15} + x^{12} + x^{10} + x^9 + x^7 + x^6 + x^5 \mod x^5 + x^2 + x + 1$
		   Anmerkung: Werte in Klammern, dienen als Platzhalter. Sie kommen eigentlich nicht vor.
		   \begin{align*}
			&x^{16} &+ x^{15} 	&+ (x^{14})	&+ (x^{13})	&+ x^{12} 	&+ (x^{11}) &+ x^{10} 	&+ x^9 	&+ (x^8)	&+ x^7	&+ x^6	&+ x^5 	&\div &x^5 &+ x^2 &+ x &+ 1 \\
			&		&			&			&			&			&			&			&		&			&		&		&		&= &x^{11} &+ x^{10} &+ x^{8} &+ x^{7} + x^{5} + x^{4} + x^{2} + x + 1\\
		  - &(x^{16} &+ (x^{15})	&+ (x^{14})	&+ x^{13} 	&+ x^{12}  	&+ x^{11})\\
			& 		&+ x^{15} 	&+ (x^{14})	&+ x^{13}	&+ (x^{12})	&+ x^{11}\\
			&-		&(+ x^{15}	&+ (x^{14})	&+ (x^{13})	&+ x^{12} 	&+ x^{11}	&+ x^{10})\\
			&		&			&			&+ x^{13}	&+ x^{12}	&+ (x^{11})	&+ (x^{10})\\
			&		& 			&-			&(+ x^{13}	&+ (x^{12})	&+ (x^{11})	&+ x^{10} 	&+ x^9 	&+ x^8)\\
			&		&			&			&			&+ x^{12}	&+ (x^{11})	&+ x^{10}	&+ (x^9)&+ x^8\\
			&		&			&			&-			&(+ x^{12}	&+ (x^{11})	&+ (x^{10})	&+ x^9 	&+ x^8		&+ x^7)\\
			&		&			&			&			&			&			&+ x^{10}	&+ x^9	&+ (x^8)	&+ (x^7)\\
			&		&			&			&			&			&-			&(+ x^{10}	&+ (x^9)&+ (x^8)	&+ x^7	&+ x^6	&+ x^5)\\
			&		&			&			&			&			&			&			&+ x^9	&+ (x^8)	&+ x^7	&+ (x^6)&+ (x^5)\\
			&		&			&			&			&			&			&-			&(+ x^9	&+ (x^8)	&+ (x^7)&+ x^6	&+ x^5		&+ x^4)\\
			&		&			&			&			&			&			&			&					&+ x^7	&+ x^6	&+ x^5		&+ x^4\\
			&		&			&			&			&			&			&			&		&-			&(+ x^7	&+ (x^6)&+ (x^5)	&+ x^4		&+ x^3 	&+ x^2)\\
			&		&			&			&			&			&			&			&		&			&		&+ x^6	&+ x^5		&+ (x^4)	&+ x^3 	&+ x^2\\
			&		&			&			&			&			&			&			&		&			&-		&(+ x^6	&+ (x^5) 	&+ (x^4)	&+ x^3	&+ x^2	&+ x^1)\\
			&		&			&			&			&			&			&			&		&			&		&		&+ x^5		&+ (x^4)	&+ (x^3)&+ (x^2)&+ x^1 & \dots
			\end{align*}
			\dots
			\begin{align*}
					&+ x^5		&+ (x^4)	&+ (x^3)&+ (x^2)	&+ x^1\\
			-		&(+ x^5 		&+ (x^4)	&+ (x^3)&+ x^2		&+ x^1 	&+ 1)\\
					&			&			&		&+ x^2 		&+ (x^1)&+ 1 &\rightarrow R(x) = x^2 + 1
		  \end{align*}
		\end{solutionordottedlines}
		
		\part Was passiert, wenn $M(x)$ falsch übertragen wird? Geben Sie dazu ein Beispiel an.
		\begin{solutionordottedlines}[2cm]
		Wenn $M(x)$ falsch übertragen wird, stimmt der Rest nicht mehr. Fehlt beispielsweise der Term $x^5$ (d.h. das letzte Bit der Ursprünglichen Nachricht ist 0 statt 1) ändert sich
		der Rest der Division $R(x)$ zu (fortgesetzt ab Zeile ):
		\begin{align*}
			+ x^{10}	&+ x^9		&+ (x^8)	&+ (x^7)	&+ x^6		&+ (x^5)\\
			- (x^{10}	&+ (x^9)	&+ (x^8)	&+ x^7		&+ x^6		&+ x^5)\\
						&+ x^9		&+ (x^8)	&+ x^7		&+ (x^6)	&+ x^5\\
			-			&+ x^9		&+ (x^8)	&+ (x^7)	&+ x^6		&+ x^5		&+ x^4)\\
						&			&			&+ x^7		&+ x^6		&+ (x^5)	&+ x^4\\
						&			&-			&(+ x^7		&+ (x^6)	&+ (x^5)	&+ x^4		&+ x^3 		&+ x^2)\\
						&			&			&			&+ x^6		&+ (x^5)	&+ (x^4)	&+ x^3 		&+ x^2\\
						&			&			&-			&(+ x^6		&+ (x^5) 	&+ (x^4)	&+ x^3		&+ x^2		&+ x^1)\\
						&			&			&			&			&+ (x^5)	&+ (x^4)	&+ (x^3)	&+ (x^2)	&+ x^1\\
						&			&			&			&			&			& -			&(+ x^3		&+ x^2		&+ (x^1) &+ 1) &\rightarrow R(x) = x^3 + x^2 + 1
		\end{align*}
		
		\end{solutionordottedlines}
		\end{landscape}
	\end{parts}

	%%%%%%%%%%%%%%%%%%%%%%%%%%%%%%%%%%%%%%%%%%%%%%%%%%%%

	%%%%%%%%%%%%%%%%%%%%%%%%%%%%%%%%%%%%%%%%%%%%%%%%%%%%

\end{questions}

