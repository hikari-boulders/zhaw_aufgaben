%% Einleitung

\begin{tabularx}{\textwidth}{Xr}
  Constantin Lazari, Marco Wettstein & \today\\
\end{tabularx}

%% Fragen
\begin{questions}
	\titledquestion{Das Wort $1010 001$ mit dem korrekt gesetzten \textit{Paritäts-Bit 1} (für
die gerade einfache Parität) wird bei der Übertragung durch Fehler verfälscht. Geben Sie an, ob der oder die Fehler erkannt
werden und begründen Sie Ihre Antwort kurz (Stichwort).
}
	\thequestiontitle
	\begin{parts}
		\part $1010\,0011 \Rightarrow 1010\,0\underline{1}11 \Rightarrow$
		\begin{solutionordottedlines}[2cm]
			Es handelt sich um einen 1-Bit-Fehler. Dieser kann aufgrund des Parität-Bits erkannt werden.
			Vier Einsen erheischen eine Null als Paritäts-Bit. 
		\end{solutionordottedlines}
		
		\part $1010\,0011 \Rightarrow \underline{0}010\,\underline{1}011 \Rightarrow $
		\begin{solutionordottedlines}[2cm]
			Es handelt sich um einen 2-Bit-Fehler. Dieser kann mit der einfachen Parität nicht erkannt werden,
			weil sich die Fehler gegenseitig maskieren.
		\end{solutionordottedlines}

		\part $1010\,0011 \Rightarrow 10\underline{0}0\,\underline{1}011 \Rightarrow $
		\begin{solutionordottedlines}[2cm]
			Es handelt sich um einen 2-Bit-Fehler. Dieser kann mit der einfachen Parität nicht erkannt werden,
			weil sich die Fehler gegenseitig maskieren. 
		\end{solutionordottedlines}
		
	\end{parts}

	%%%%%%%%%%%%%%%%%%%%%%%%%%%%%%%%%%%%%%%%%%%%%%%%%%%%
	\question Die Daten werden mit einer 2-Dimensionalen Parität (gerade) gesichert:
	\begin{itemize}
		\item das 1. Paritäts-Bit entspricht der einfachen Parität für jeweils einem 7-Bit-Wort
		\item das 2. Paritäts-Bit ist das jeweils i-te Bit (Spalte)
	\end{itemize}
	Gegeben sind drei übertragene Blöcke mit Übertragungsfehlern.
	Werden diese erkannt? Begründen Sie Ihre Antwort.

	\begin{parts}
		\part
		\begin{tabular}{|c|c|c|c|c|c|c|c|}
		\hline
			1 & 1 & 0 & 0 & 0 & 0 & 1 & \textbf{1} \\\hline %1
			1 & 1 & 0 & 0 & 0 & 1 & 0 & \textbf{1} \\\hline %2
			1 & 1 & 0 & 1 & 0 & 1 & 1 & \textbf{0} \\\hline %3
			1 & 1 & 0 & 0 & 1 & 0 & 0 & \textbf{1} \\\hline %4
			1 & 1 & 0 & 0 & 1 & 0 & 1 & \textbf{0} \\\hline %5
			1 & 1 & 0 & 1 & 1 & 1 & 0 & \textbf{0} \\\hline %6
			1 & 1 & 0 & 0 & 1 & 1 & 1 & \textbf{1} \\\hline %7
			1 & 1 & 0 & 1 & 0 & 0 & 0 & \textbf{1} \\\hline %8
			\textbf{0} & \textbf{0} & \textbf{0} & \textbf{1} & \textbf{0} & \textbf{0} & \textbf{0} & \textbf{1}\\\hline
		\end{tabular}

		\begin{solutionordottedlines}[2cm]
			Erkannt wird ein nicht korrigierbarer 2-Bit-Fehler:
			\begin{enumerate}
				\item Zeile 3 enthält fünf Einsen, Paritätsbit müsste 1 sein.
				\item Zeile 6 enthält fünf Einsen, Paritätsbit müsste 1 sein. 
			\end{enumerate}
		\end{solutionordottedlines}
		\pagebreak

		\part
		\begin{tabular}{|c|c|c|c|c|c|c|c|}
		\hline
			1 & 1 & 0 & 0 & 0 & 0 & 1 & \textbf{1} \\\hline %1
			1 & 1 & 0 & 0 & 0 & 1 & 0 & \textbf{1} \\\hline %2
			1 & 1 & 0 & 1 & 0 & 1 & 1 & \textbf{0} \\\hline %3
			1 & 1 & 0 & 0 & 1 & 0 & 0 & \textbf{1} \\\hline %4
			1 & 1 & 0 & 0 & 1 & 1 & 1 & \textbf{0} \\\hline %5
			1 & 1 & 0 & 0 & 1 & 1 & 0 & \textbf{0} \\\hline %6
			1 & 1 & 1 & 0 & 1 & 1 & 1 & \textbf{1} \\\hline %7
			1 & 1 & 0 & 1 & 0 & 0 & 0 & \textbf{1} \\\hline %8
			\textbf{0} & \textbf{0} & \textbf{0} & \textbf{1} & \textbf{0} & \textbf{0} & \textbf{0} & \textbf{1}\\\hline
		\end{tabular}

		\begin{solutionordottedlines}[2cm]
		  Erkannt wird ein nicht korrigierbarer Mehr-Bit-Fehler:
		  \begin{enumerate}
		   \item Zeile 3 enthält fünf Einsen, Paritätsbit müsste 1 sein.
		   \item Zeile 5 enthält fünf Einsen, Paritätsbit müsste 1 sein.
		   \item Zeile 7 enthält sechs Einsen, Paritätsbit müsste 0 sein.
		   \item Spalte 3 enthält eine Eins, Paritätsbit müsste 1 sein.
		   \item Spalte 4 enthält 2 Einsen, Paritätsbit müsste 0 sein.
		   \item Spalte 6 enthält 5 Einsen, Paritätsbit müsste 1 sein.
		  \end{enumerate}
		\end{solutionordottedlines}

		\part
		\begin{tabular}{|c|c|c|c|c|c|c|c|}
		\hline
			1 & 1 & 0 & 0 & 0 & 0 & 1 & \textbf{1} \\\hline %1
			1 & 1 & 0 & 0 & 0 & 1 & 0 & \textbf{1} \\\hline %2
			1 & 1 & 0 & 0 & 0 & 1 & 1 & \textbf{0} \\\hline %3
			1 & 1 & 0 & 0 & 1 & 0 & 0 & \textbf{1} \\\hline %4
			1 & 1 & 0 & 0 & 1 & 0 & 1 & \textbf{0} \\\hline %5
			1 & 1 & 0 & 0 & 1 & 1 & 0 & \textbf{0} \\\hline %6
			1 & 1 & 0 & 0 & 1 & 1 & 1 & \textbf{1} \\\hline %7
			1 & 1 & 0 & 1 & 0 & 0 & 0 & \textbf{1} \\\hline %8
			\textbf{0} & \textbf{0} & \textbf{0} & \textbf{1} & \textbf{0} & \textbf{0} & \textbf{0} & \textbf{1}\\\hline
		\end{tabular}

		\begin{solutionordottedlines}[2cm]
		  In diesem Fall sind keine Fehler erkennbar, weil die Paritätsbits alle stimmen.
		  Daraus kann leider nicht gefolgert werden, dass keine Fehler da sind.
		  Beispielsweise bei einer Anordnung im Viereck, könnte der Block mindestens 4 Einzelbitfehler aufweisen.
		\end{solutionordottedlines}

		
	\end{parts}

	%%%%%%%%%%%%%%%%%%%%%%%%%%%%%%%%%%%%%%%%%%%%%%%%%%%%
	\begin{landscape}
	\question Gegeben ist:
	\begin{itemize}
		\item das Generatorpolynom $g = x^5 + x^2 + x + 1$
		\item eine zu übertragende Bitfolge: $1101\,0111\,0111$
	\end{itemize}


	\begin{parts}
		\part Geben Sie $I(x)$ an
		\begin{solutionordottedlines}[2cm]
		\begin{equation*}
		 I(x) = x^{11} + x^{10} + x^8 + x^6 + x^5 + x^4 + x^2 + x + 1
		\end{equation*}

		\end{solutionordottedlines}
		\part Berechnen Sie $M(x)$ sowie $R(x)$ mit dem CRC-Verfahren
		\begin{solutionordottedlines}[2cm]
		  \begin{equation*}
		   M(x) = (x^{11} + x^{10} + 	x^8 + 		x^6 + 		x^5 	+ 	x^4 + x^2 + x 	+ 	1) \cdot x^5 
				= x^{16} + 	x^{15} + 	x^{13} + 	x^{11} + 	x^{10} + 	x^9 + x^7 + x^6 +	x^5
		   \end{equation*}
			Der Rest ergibt sich aus der Polynom-Division von $R(x) = M(x) \mod g$:
			\begin{equation*}
		   \begin{array}{ccccccccccccccccccc}
			&(x^{16} 	&+ x^{15}	& 		&+ x^{13}	& 			&+ x^{11}	&+ x^{10} 	&+ x^9 	&			&+ x^7		&+ x^6		&+ x^5) &\multicolumn{6}{l}{\mod (x^5 + x^2 + x + 1)} \\ % = x^{11} + x^{10} + x^{6} + x^4 + x^3 + 1
		  - &(x^{16} 	&			&		&+ x^{13} 	&+ x^{12}  	&+ x^{11})	&			&		&			&			&			&		&		&		&		&		&		&\\
			\cline{2-7}			
			& 			&x^{15} 	&		&			&+ x^{12}	&			&			&		&			&			&			&		&		&		&		&		&		&\\
			&-			&(x^{15}	&		&			&+ x^{12} 	&+ x^{11}	&+ x^{10})	&		&			&			&			&		&		&		&		&		&		&\\
			\cline{3-8}
			&			&			&		&			&			&x^{11}		&			&		&			&			&			&		&		&		&		&		&		&\\
			&			& 			&		&			&-			&(x^{11}	& 			& 		&+x^8		&+ x^7		&+ x^6)		&		&		&		&		&		&		&\\
			\cline{7-12}
			&			&			&		&			&			&			&			&x^9	&+ x^8 		& 			&			&+ x^5	&		&		&		&		&		&\\
			&			&			&		&			&			&			&-			&(x^9	&			&			&+ x^6		&+ x^5 	&+ x^4)	&		&		&		&		&\\
			\cline{9-14}
			&			&			&		&			&			&			&			&		&x^8		&			&+ x^6		&		&+ x^4	&		&		&		& 		&\\
			&			&			&		&			&			&			&			&-		&(x^8		&			&			&+ x^5	&+ x^4	&+ x^3	&		&		&		&\\
			\cline{10-15}
			&			&			&		&			&			&			&			&		&			&			&x^6		&+ x^5	&		&+ x^3 	&		&		&		&\\
			&			&			&		&			&			&			&			&		&			&-			&(x^6		&		&		&+ x^3	&+ x^2	& +x	&		&\\	
			\cline{12-17}
			&			&			&		&			&			&			&			&		&			&			&			&x^5	&		&		&+ x^2	& +x	&		&\\
			&			&			&		&			&			&			&			&		&			&			&-			&(x^5	&		&		&+ x^2	& +x	& +1)	&\\
			\cline{13-18}
			&			&			&		&			&			&			&			&		&			&			&			&		&		&		&		&		& 1 	&\rightarrow R(x) = 1\\
			\end{array}
			\end{equation*}
		\end{solutionordottedlines}
		
		\part Was passiert, wenn $M(x)$ falsch übertragen wird? Geben Sie dazu ein Beispiel an.
		\begin{solutionordottedlines}[2cm]
		Übertragen wird $M(x) + R(x)$. Da bei der Berechnung die Subtraktion der Addition entspricht, muss die Polynom Division $(M(x) + R(x)) \mod g$ den Rest null ergeben.
		Wird $M(x)$ falsch übertragen, ist in der überwiegenden Mehrheit der Fälle, die Ausnahmen sind alle Fälle, in denen durch den Fehler ein vielfaches von $g$ entsteht, der Rest nicht mehr null. 
		Fehlt beispielsweise der Term $x^5$ (d.h. das letzte Bit der Ursprünglichen Nachricht ist 0 statt 1) ändert sich der Rest der Division $R(x)$ wie folgt:
			\begin{equation*}
		   \begin{array}{ccccccccccccccccccc}
			&(x^{16} 	&+ x^{15}	& 		&+ x^{13}	& 			&+ x^{11}	&+ x^{10} 	&+ x^9 	&			&+ x^7		&+ x^6		& 		&		&		&		&		&+1		&\mod g \\
		  - &(x^{16} 	&			&		&+ x^{13} 	&+ x^{12}  	&+ x^{11})	&			&		&			&			&			&		&		&		&		&		&		&\\
			\cline{2-7}			
			& 			&x^{15} 	&		&			&+ x^{12}	&			&			&		&			&			&			&		&		&		&		&		&		&\\
			&-			&(x^{15}	&		&			&+ x^{12} 	&+ x^{11}	&+ x^{10})	&		&			&			&			&		&		&		&		&		&		&\\
			\cline{3-8}
			&			&			&		&			&			&x^{11}		&			&		&			&			&			&		&		&		&		&		&		&\\
			&			& 			&		&			&-			&(x^{11}	& 			& 		&+x^8		&+ x^7		&+ x^6)		&		&		&		&		&		&		&\\
			\cline{7-12}
			&			&			&		&			&			&			&			&x^9	&+ x^8 		& 			&			&		&		&		&		&		&		&\\
			&			&			&		&			&			&			&-			&(x^9	&			&			&+ x^6		&+ x^5 	&+ x^4)	&		&		&		&		&\\
			\cline{9-14}
			&			&			&		&			&			&			&			&		&x^8		&			&+ x^6		&+ x^5	&+ x^4	&		&		&		& 		&\\
			&			&			&		&			&			&			&			&-		&(x^8		&			&			&+ x^5	&+ x^4	&+ x^3	&		&		&		&\\
			\cline{10-15}
			&			&			&		&			&			&			&			&		&			&			&x^6		&		&		&+ x^3 	&		&		&		&\\
			&			&			&		&			&			&			&			&		&			&-			&(x^6		&		&		&+ x^3	&+ x^2	& +x	&		&\\	
			\cline{12-18}
			&			&			&		&			&			&			&			&		&			&			&			&		&		&		&x^2	& +x	&+1		&\\
			\end{array}
			\end{equation*}
			Der Rest ist also $R(x) = x^2 + x + 1 \neq 0 \rightarrow$ bei der Übertragung ist ein Fehler aufgetreten.
		\end{solutionordottedlines}
	\end{parts}
	\end{landscape}

	%%%%%%%%%%%%%%%%%%%%%%%%%%%%%%%%%%%%%%%%%%%%%%%%%%%%


	%%%%%%%%%%%%%%%%%%%%%%%%%%%%%%%%%%%%%%%%%%%%%%%%%%%%

\end{questions}

