\pagebreak
\question
Der Befehlssatz für den „Mini-Power-PC“ ist sehr klein / eingeschränkt. Sie
haben gelernt, dass Zugriffe auf den Arbeitsspeicher sehr langsam sind
(zum Teil deutlich mehr als 100 Zyklen).
\begin{parts}
\part
Welchen Befehlstyp würden Sie auf jeden Fall ergänzen, um die Code-Ausführung erheblich zu beschleunigen?
\begin{solutionordottedlines}[2cm]
Befehle, um häufig benötigte Werte temporär zwischenzulagern, also z.\,B. in ein unbenutztes Register zu verschieben (Register $a \rightarrow$ Register $b$).
\end{solutionordottedlines}
%%% Next subquestion

\part
Kann mit dem Befehlssatz ein Arbeitsspeicher von 16 KiB genutzt werden? Antwort bitte begründen.
\begin{solutionordottedlines}[2cm]
Nein, den Befehlen \texttt{LWDD} und \texttt{SWDD} stehen nur 10 Bit zur Adressierung des Speichers zur Verfügung. 10 Bit erlauben als nur 1\,024 Speicheradressen.

Mittels indirekter Adressierung (z.B. via eines 16-Bit Registers), wäre es möglich. Der Befehle müsste dann allerdings von der Form LWDD Rin, Rout sein. Gleiches gilt für SWDD. 
% Solution goes here
\end{solutionordottedlines}
%%% Next subquestion

\end{parts}