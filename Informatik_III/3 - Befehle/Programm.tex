\question
Gegeben sei der Befehlssatz für den \enquote{Mini-Power-PC}. Die Aufgabe Summe = $a + 4 * b + 8 * c$ soll über ein Programm für den \enquote{Mini-Power-PC} berechnet werden.
\begin{parts}
\part
Schreiben Sie den Programm-Code mit mnemonische Symbolen (in Assembler) 
\begin{solutionordottedlines}[2cm]
100 LWDD R0 \#202; Load b into Akku\\ 
101 SLA; Multiply by 2\\
102 BCD 119; Jump to end, on overflow\\
103 SLA; Multiply by 2 -- So multiplied by 4\\
104 BCD 119; Jump to end on overflow\\
105 LWDD R1 \#200; Load a into Akku\\
106 ADD R1; Add a to 4 * b\\
107 SWDD R0 \#206; Store result as d\\
108 LWDD R0 \#202; Load c\\
109 SLA; Multiply by 2\\
110 BCD 119; Jump to end on overflow\\
111 SLA; Multiply by 2 -- So multiplied by 4\\
112 BCD 119; Jump to end on overflow\\
113 SLA; Multiply by 2 -- So multiplied by 8\\
114 BCD 119; Jump to end on overflow\\
115 LWDD R1 \#206; Load d in Register 1\\
116 ADD R1; Add d to 8 * c\\
117 BCD 118; Jump to end on overflow\\
118 SWDD R0 \#206; Write result to 206\\
119 END; Program finishes
\end{solutionordottedlines}
%%% Next subquestion

\part
Übersetzen Sie den Code in Maschinencode
\begin{solutionordottedlines}[2cm]
Das würde der \enquote{Mini-Power-PC} machen, er ist aber noch nicht ganz fertig.
\end{solutionordottedlines}
%%% Next subquestion

\part
Berechnen Sie mit Hilfe Ihres Programms:
\begin{itemize}
	\item Summe für $a = 14, b = 7$ und $c = 66$
	\item Summe für $a = 25, b = -14$ und $c = -123$
	\item Summe für $a = -125, b = 10\,000$ und $c = 16$
	\item Summe für $a = 1\,000, b = 10\,000$ und $c = -2\,000$
\end{itemize}
Anmerkungen:
\begin{itemize}
	\item Das Programm beginnt in der Speicherzelle 100, die Variablen a, b und c liegen in den Speicherzellen 200 / 201 (für a), 202 / 203 (für b) und 204 / 205 (für c).
	\item Am Ende des Programm soll die Summe in der Speicherzelle 206 / 207 stehen. Bei einem Überlauf soll das Programm abgebrochen werden.
	\item Sie können selbstverständlich den Emulator der Aufgabenserie 3b für die Verifizierung und Berechnung nutzen - er könnte auch als \enquote{Debugger} sehr hilfreich sein \dots
\end{itemize}


\begin{solutionordottedlines}[2cm]
Summe würde der \enquote{Mini-Power-PC} berechnen, er ist aber noch nicht ganz fertig.
\begin{itemize}
	\item Summe für $a = 14, b = 7$ und $c = 66$
	\item Summe für $a = 25, b = -14$ und $c = -123$ 
	\item Summe für $a = -125, b = 10\,000$ und $c = 16$
	\item Summe für $a = 1\,000, b = 10\,000$ und $c = -2\,000$
\end{itemize}
\end{solutionordottedlines}

\end{parts}