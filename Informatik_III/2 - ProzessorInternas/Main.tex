\documentclass[12pt, a4paper, answers]{exam} % Lösungen
%\documentclass[12pt, addpoints, a4paper]{exam} % Ohne Lösungen
\newcommand{\myAuthor}{Constantin Lazari, Marco Wettstein}
\newcommand{\myTitle}{Übungen Informatik}
\newcommand{\mySubject}{Informatik III (2013)}
\newcommand{\myNumber}{2}

\usepackage[utf8]{inputenc}
\usepackage[pdftex]{graphicx} 
\usepackage{microtype}
\usepackage[pdfborder={0 0 0}, plainpages=false, pdfpagelabels]{hyperref} 
\usepackage[ngerman]{babel}
\usepackage[babel]{csquotes}
\usepackage{tabularx} 

\usepackage{tikz}
\usetikzlibrary{arrows,automata}
\usepackage{amsmath,amssymb,amsthm}

\usepackage{lmodern} %Type1-Schriftart fuer nicht-englische Texte
\hyphenation{eine einer eines} % Trennung von eine, einer, eines vermeiden
\usepackage{microtype}

\usepackage{color}
\usepackage{stmaryrd}
\usepackage{booktabs}


%% Listings %%%%%%%%%%%%%%%%%%%%%%%%%%%%%%%%%%%%%%%%%%%%%%%%%
%\usepackage{verbatim}
\usepackage{listings}
\lstloadlanguages{[LaTeX]TeX}
{\lstset{%
  basicstyle=\footnotesize\ttfamily,
  commentstyle=\slshape\color{green!50!black},
  keywordstyle=\bfseries\color{blue!50!black},
  identifierstyle=\color{blue},
  stringstyle=\color{orange},
  escapechar=\#,
  emphstyle=\color{red}}
}
{
  \lstset{%
    basicstyle=\ttfamily,
    keywordstyle=\bfseries,
    commentstyle=\itshape,
    escapechar=\#,
    emphstyle=\bfseries\color{red}
  }
}

\hypersetup{
	pdfauthor   = {\myAuthor},
	pdftitle    = {\myTitle},
	pdfsubject  = {\mySubject},
	pdfkeywords = {},
	pdfcreator  = {Kile},
	pdfproducer = {pdflatex},
	colorlinks 	= false
} 

\setlength{\parindent}{0em}
\setlength{\parskip}{0.75em}

%% Exam Settings
\pagestyle{headandfoot}
%\firstpageheader{Benutzer/innen im Umgang mit Informatikmitteln instruieren}{}{Lernkontrolle 1}
\firstpageheader{\mySubject}{}{Übung \myNumber}
\firstpageheadrule

%\runningheader{Benutzer/innen im Umgang mit Informatikmitteln instruieren}{}{Lernkontrolle 1}
\runningheader{\mySubject}{}{
\ifprintanswers
  Lösung Übung \myNumber
\else
  Übung \myNumber
\fi
}
\runningheadrule

\firstpagefooter{}{Seite \thepage\ von \numpages}{}
\firstpagefootrule

\runningfooter{}{Seite \thepage\ von \numpages}{}
\runningfootrule

\pointsinrightmargin
\pointpoints{Punkt}{Punkte}
\bonuspointpoints{Bonuspunkt}{Bonuspunkte}
\renewcommand{\solutiontitle}{\noindent\textbf{Lösung:}\par\noindent}

\CorrectChoiceEmphasis{\bfseries}
\renewcommand\choicelabel{$\boxempty$}

\begin{document}
	\begin{tabularx}{\textwidth}{Xr}
	\myAuthor & \today\\
	\end{tabularx}
  %% Lernkontrolle einbinden
	\begin{questions}
	\question
% question title goes here
Gegeben sei ein Prozessor mit einer Taktzykluszeit von 1.25 GHz und einem CPI-Wert von 1.45 (der Prozessor verfügt über keine Pipeline). Ein Programm benötigt zur Ausführung $150\,000$ Befehle.
\begin{parts}
\part Wie lang ist die ungefähre Ausführungszeit des Programms?
\begin{solutionordottedlines}[2cm]
% Solution goes here
\begin{equation*}
	\mbox{Taktzyklen: }z = 150\,000 \mbox{ Befehl} \cdot 1.45 \frac{\mbox{Taktzyklus}}{\mbox{Befehl}} = 217\,500 \mbox{ Taktzyklus}
\end{equation*}
\begin{equation*}
	\mbox{Zeit: } t = \frac{217\,500 \mbox{ Taktzyklus}}{1.25 \cdot 10^9 \frac{\mbox{Taktzyklus}}{\mbox{s}}} = 0.000174 \mbox{ s} = 174~\mu \mbox{s}
\end{equation*}
Die Ausführungszeit beträgt $174~\mu$s
\end{solutionordottedlines}
%%% Next subquestion

\part Wieso ist der berechnete Wert nur ein Näherungswert?
% Subquestion text goes here
\begin{solutionordottedlines}[2cm]
Der CPI-Wert ist ein geschätzter Mittelwert. Er kann je nach Komplixität der verwendeten Befehle grösser oder kleiner sein. Der CPI-Wert ist also nicht exakt $\Rightarrow$ berechnete Werte sind ebenfalls exakt.
\end{solutionordottedlines}
%%% Next subquestion

\part Der Prozessor wird durch einen leistungsfähigeren Prozessor mit 0.4 ns Taktzykluszeit und einem CPI-Wert von 1.8 ersetzt. Wie lang ist nun die Ausführungszeit des Programms?
\begin{solutionordottedlines}[2cm]
% Solution goes here
\begin{equation*}
	\mbox{Taktzyklen: } z = 150\,000 \mbox{ Befehl} \cdot 1.8 \frac{\mbox{Taktzyklus}}{\mbox{Befehl}} = 270\,000 \mbox{ Taktzyklus}
\end{equation*}
\begin{equation*}
	\mbox{Zeit: }t = 270\,000 \mbox{ Taktzyklus} \cdot 0.4 \cdot 10^{-9} \frac{\mbox{s}}{\mbox{Taktzyklus}} = 0.000108 \mbox{ s} = 108~\mu \mbox{s}
\end{equation*}
Die Ausführungszeit beträgt $108~\mu$s
\end{solutionordottedlines}

\part Der Prozessor (von c) wird um 10\% übertaktet (\enquote{overclocking}). Die erzielte Leistungssteigerung beträgt in der Realität aber nur knapp
5\%. Wieso?
\begin{solutionordottedlines}
Die Rechengeschwindigkeit hängt nicht allein von der Taktrate ab. Insbesondere die Datenübertragungsbusse spielen eine wichtige Rolle. Sie werden aber nicht mit-übertaktet.
\end{solutionordottedlines}
\end{parts}
\pagebreak
	\question
Gegeben sei ein einfacher Prozessor ohne Pipelining mit einer Wortbreite von 2 Byte (für Daten und Befehle). 
\begin{parts}
\part
Welchen Wert beinhaltet der Befehlszähler jeweils nach Ausführung
der jeweiligen Befehle der folgenden Befehlssequenz (der Initialwert
sei $24\,048$ für den ersten Befehl): Ladebefehl, Ladebefehl, Addition, unbedingter Sprung um -12, Speicherbefehl, unbedingter Sprung um 
+8, Addition \dots?
\begin{solutionordottedlines}[2cm]
\begin{center}
\begin{tabular}{lrl}
	\toprule
	\textbf{Befehl} & \textbf{Befehlszähler} & \textbf{Kommentar}\\\midrule
	Ladebefehl				& $24\,048$ & Initialwert\\\hline
	Ladebefehl				& $24\,050$ & $+ 2$\\\hline
	Additionsbefehl			& $24\,052$ & $+ 2$\\\hline
	Sprungbefehl 			& $24\,054$ & $+ 2$\\\hline
	\textit{Sprung}			& $24\,042$ & $-12$\\\hline
	Speicherbefehl			& $24\,042$ & $+ 2$\\\hline
	Sprungbefehl 			& $24\,044$ & $+ 2$\\\hline
	\textit{Sprung}			& $24\,052$ & $+ 8$\\\hline
	Additionsbefehl			& $24\,052$ & $+ 2$\\\hline
	\dots					& $24\,054$ & $+ 2$\\\bottomrule
\end{tabular}

\end{center}
\end{solutionordottedlines}
%%% Next subquestion

\part Was sehen Sie als Informatiker sofort?
\begin{solutionordottedlines}[2cm]
An Stelle $24\,054$ steht wieder der unbedingte erste Sprungbefehl (zurück auf Feld $24\,042$). Das Programm wird also über diese Zeile nicht hinaus kommen. Es ist in einer Endlosschleife gefangen.
\end{solutionordottedlines}
%%% Next subquestion

\end{parts}


	\question
Gegeben sei ein Prozessor mit 4-stufiger Pipeline (die vier Stufen, wie in der Vorlesung angegeben) und folgender Ausschnitt einer Programmabfolge: 

\dots, Load, Sprung, Addition, ODER-Operation, Store, Subtraktion, Sprung, AND-Operation, \dots
\begin{parts}
\part
Skizzieren Sie graphisch eine (mögliche) Ausführungsabfolge, unter der Annahme, das beim 1. Sprung zu einer nicht vorhergesehenen Adresse gesprungen wird (\enquote{branch prediction} war falsch).
\begin{solutionordottedlines}[2cm]
% Solution goes here
siehe https://docs.google.com/spreadsheet/ccc?key=0AnuHoJusdbkydFB2ZjA1WDllQVl0V1BVY3A2d1M4bFE&usp=sharing

evtl. als pdf exportieren und in latex inkludieren
\end{solutionordottedlines}
%%% Next subquestion

\part
Beschreiben Sie in Ihren Worten, was ein \enquote{pipeline flush} bedeutet.
\begin{solutionordottedlines}[2cm]
% Solution goes here
Bei einem Pipeline Flush werden Ergebnisse von abgearbeiteten Teilschritten in der Pipeline verworfen, die Pipeline wird geleert und nach dem letzten gültigen Befehl wieder gefüllt.
\end{solutionordottedlines}
%%% Next subquestion

\end{parts}
	\question
Eine effektive Möglichkeit der Leistungssteigerung bei Prozessoren ist Pipelining.
\begin{parts}
\part Begründen Sie, warum eine n-stufige Pipeline nicht automatisch zu einer n-fachen Leistungssteigerung führt, selbst wenn es gelingt, die Zykluszeit auf 1/$n$ zu reduzieren (\enquote{perfekte Gleichverteilung} der Stufen -- in der Praxis eigentlich nicht realisierbar).
\begin{solutionordottedlines}[2cm]
% Solution goes here
\end{solutionordottedlines}
%%% Next subquestion
\end{parts}
	\question
Gegeben sei ein Prozessor ohne Pipeline mit der „bekannten“ Befehlsabarbeitung (siehe Vorlesung) und einer Zykluszeit von 20 MHz. Ein Analyse hat ergeben, dass die einzelnen Teilschritte sehr unterschiedliche Zeit erfordern:

z.\,B. \enquote{Befehl laden} $\leq 10$ ns, 
\enquote{Register lesen} $\leq 3$ ns, 
\enquote{Rechenoperation durchführen} $\leq 5$ ns, 
\enquote{Speicherzugriff} $\leq 20$ ns und 
\enquote{Register schreiben} $\leq 5$ ns, \dots 

Sie implementieren denselben Prozessor mit einer 5-stufigen Pipeline
(die bisherigen Teilschritte erfordern gleich viel Zeit).
\begin{parts}
\part
Wie gross ist die Zykluszeit des neuen Prozessors?
\begin{solutionordottedlines}[2cm]
% Solution goes here
100 MHz, pro Zyklus wird aber nur ein Teilschritt des Befehls abgerbeitet.
\end{solutionordottedlines}
%%% Next subquestion

\part
Um wie viel schneller wird nun ein Befehl maximal ausgeführt?
\begin{solutionordottedlines}[2cm]
% Solution goes here
Ein Befehl wird gleich schnell ausgeführt. 
\end{solutionordottedlines}
%%% Next subquestion

\part
Um wie viel schneller wird ein Programm maximal ausgeführt?
\begin{solutionordottedlines}[2cm]
% Solution goes here
Wenn ein Programm n Befehle hat, ist das Programm auf der CPU mit der 5-stufigen Pipeline maximal (5 + n-1 ) / (5 * n) mal schneller als auf der CPU ohne pipeline.
\end{solutionordottedlines}
%%% Next subquestion

\part
Wie könnte eine \enquote{bessere} Pipeline-Struktur entwickelt werden?
\begin{solutionordottedlines}[2cm]
% Solution goes here
Lange Pipelines sind einerseits komplexer im Aufbau, anderseits anfällig für Pipeline Konflikte (.b. Control Hazards). In einem solchen fall muss die Pipeline geleert werden, was die Ausführung verlangsamt.

Dem kann man einerseits mit besseren Sprungvorhersagen und anderseits kürzeren Pipelines begegnen.

Weniger und ähnlich lange Befehle ist eine weitere Optimierung, wie sie in RISC CPUs gemacht wird.
\end{solutionordottedlines}
%%% Next subquestion
\end{parts}
	\end{questions}
\end{document}