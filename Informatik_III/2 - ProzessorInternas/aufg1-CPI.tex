\question
% question title goes here
Gegeben sei ein Prozessor mit einer Taktzykluszeit von 1.25 GHz und einem CPI-Wert von 1.45 (der Prozessor verfügt über keine Pipeline). Ein Programm benötigt zur Ausführung $150\,000$ Befehle.
\begin{parts}
\part Wie lang ist die ungefähre Ausführungszeit des Programms?
\begin{solutionordottedlines}[2cm]
% Solution goes here
\begin{equation*}
	\mbox{Taktzyklen: }z = 150\,000 \mbox{ Befehl} \cdot 1.45 \frac{\mbox{Taktzyklus}}{\mbox{Befehl}} = 217\,500 \mbox{ Taktzyklus}
\end{equation*}
\begin{equation*}
	\mbox{Zeit: } t = \frac{217\,500 \mbox{ Taktzyklus}}{1.25 \cdot 10^9 \frac{\mbox{Taktzyklus}}{\mbox{s}}} = 0.000174 \mbox{ s} = 174~\mu \mbox{s}
\end{equation*}
Die Ausführungszeit beträgt $174~\mu$s
\end{solutionordottedlines}
%%% Next subquestion

\part Wieso ist der berechnete Wert nur ein Näherungswert?
% Subquestion text goes here
\begin{solutionordottedlines}[2cm]
Der CPI-Wert ist ein geschätzter Mittelwert. Er kann je nach Komplixität der verwendeten Befehle grösser oder kleiner sein. Der CPI-Wert ist also nicht exakt $\Rightarrow$ berechnete Werte sind ebenfalls exakt.
\end{solutionordottedlines}
%%% Next subquestion

\part Der Prozessor wird durch einen leistungsfähigeren Prozessor mit 0.4 ns Taktzykluszeit und einem CPI-Wert von 1.8 ersetzt. Wie lang ist nun die Ausführungszeit des Programms?
\begin{solutionordottedlines}[2cm]
% Solution goes here
\begin{equation*}
	\mbox{Taktzyklen: } z = 150\,000 \mbox{ Befehl} \cdot 1.8 \frac{\mbox{Taktzyklus}}{\mbox{Befehl}} = 270\,000 \mbox{ Taktzyklus}
\end{equation*}
\begin{equation*}
	\mbox{Zeit: }t = 270\,000 \mbox{ Taktzyklus} \cdot 0.4 \cdot 10^{-9} \frac{\mbox{s}}{\mbox{Taktzyklus}} = 0.000108 \mbox{ s} = 108~\mu \mbox{s}
\end{equation*}
Die Ausführungszeit beträgt $108~\mu$s
\end{solutionordottedlines}

\part Der Prozessor (von c) wird um 10\% übertaktet (\enquote{overclocking}). Die erzielte Leistungssteigerung beträgt in der Realität aber nur knapp
5\%. Wieso?
\begin{solutionordottedlines}
Die Rechengeschwindigkeit hängt nicht allein von der Taktrate ab. Insbesondere die Datenübertragungsbusse spielen eine wichtige Rolle. Sie werden aber nicht mit-übertaktet.
\end{solutionordottedlines}
\end{parts}
\pagebreak