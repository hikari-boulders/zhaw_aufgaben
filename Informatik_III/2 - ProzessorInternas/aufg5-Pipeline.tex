\question
Gegeben sei ein Prozessor ohne Pipeline mit der „bekannten“ Befehlsabarbeitung (siehe Vorlesung) und einer Zykluszeit von 20 MHz. Ein Analyse hat ergeben, dass die einzelnen Teilschritte sehr unterschiedliche Zeit erfordern:

z.\,B. \enquote{Befehl laden} $\leq 10$ ns, 
\enquote{Register lesen} $\leq 3$ ns, 
\enquote{Rechenoperation durchführen} $\leq 5$ ns, 
\enquote{Speicherzugriff} $\leq 20$ ns und 
\enquote{Register schreiben} $\leq 5$ ns, \dots 

Sie implementieren denselben Prozessor mit einer 5-stufigen Pipeline
(die bisherigen Teilschritte erfordern gleich viel Zeit).
\begin{parts}
\part
Wie gross ist die Zykluszeit des neuen Prozessors?
\begin{solutionordottedlines}[2cm]
% Solution goes here
100 MHz, pro Zyklus wird aber nur ein Teilschritt des Befehls abgerbeitet.
\end{solutionordottedlines}
%%% Next subquestion

\part
Um wie viel schneller wird nun ein Befehl maximal ausgeführt?
\begin{solutionordottedlines}[2cm]
% Solution goes here
Ein Befehl wird gleich schnell ausgeführt. 
\end{solutionordottedlines}
%%% Next subquestion

\part
Um wie viel schneller wird ein Programm maximal ausgeführt?
\begin{solutionordottedlines}[2cm]
% Solution goes here
Wenn ein Programm n Befehle hat, ist das Programm auf der CPU mit der 5-stufigen Pipeline maximal (5 + n-1 ) / (5 * n) mal schneller als auf der CPU ohne pipeline.
\end{solutionordottedlines}
%%% Next subquestion

\part
Wie könnte eine \enquote{bessere} Pipeline-Struktur entwickelt werden?
\begin{solutionordottedlines}[2cm]
% Solution goes here
Lange Pipelines sind einerseits komplexer im Aufbau, anderseits anfällig für Pipeline Konflikte (.b. Control Hazards). In einem solchen fall muss die Pipeline geleert werden, was die Ausführung verlangsamt.

Dem kann man einerseits mit besseren Sprungvorhersagen und anderseits kürzeren Pipelines begegnen.

Weniger und ähnlich lange Befehle ist eine weitere Optimierung, wie sie in RISC CPUs gemacht wird.
\end{solutionordottedlines}
%%% Next subquestion
\end{parts}