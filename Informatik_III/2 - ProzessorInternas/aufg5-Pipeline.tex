\question
Gegeben sei ein Prozessor ohne Pipeline mit der „bekannten“ Befehlsabarbeitung (siehe Vorlesung) und einer Zykluszeit von 20 MHz. Ein Analyse hat ergeben, dass die einzelnen Teilschritte sehr unterschiedliche Zeit erfordern:

z.\,B. \enquote{Befehl laden} $\leq 10$ ns, 
\enquote{Register lesen} $\leq 3$ ns, 
\enquote{Rechenoperation durchführen} $\leq 5$ ns, 
\enquote{Speicherzugriff} $\leq 20$ ns und 
\enquote{Register schreiben} $\leq 5$ ns, \dots 

Sie implementieren denselben Prozessor mit einer 5-stufigen Pipeline
(die bisherigen Teilschritte erfordern gleich viel Zeit).
\begin{parts}
\part
Wie gross ist die Zykluszeit des neuen Prozessors?
\begin{solutionordottedlines}[2cm]
% Solution goes here
\end{solutionordottedlines}
%%% Next subquestion

\part
Um wie viel schneller wird nun ein Befehl maximal ausgeführt?
\begin{solutionordottedlines}[2cm]
% Solution goes here
\end{solutionordottedlines}
%%% Next subquestion

\part
Um wie viel schneller wird ein Programm maximal ausgeführt?
\begin{solutionordottedlines}[2cm]
% Solution goes here
\end{solutionordottedlines}
%%% Next subquestion

\part
Wie könnte eine \enquote{bessere} Pipeline-Struktur entwickelt werden?
\begin{solutionordottedlines}[2cm]
% Solution goes here
\end{solutionordottedlines}
%%% Next subquestion
\end{parts}