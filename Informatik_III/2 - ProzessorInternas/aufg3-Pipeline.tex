\question
Gegeben sei ein Prozessor mit 4-stufiger Pipeline (die vier Stufen, wie in der Vorlesung angegeben) und folgender Ausschnitt einer Programmabfolge: 

\dots, Load, Sprung, Addition, ODER-Operation, Store, Subtraktion, Sprung, AND-Operation, \dots
\begin{parts}
\part
Skizzieren Sie graphisch eine (mögliche) Ausführungsabfolge, unter der Annahme, das beim 1. Sprung zu einer nicht vorhergesehenen Adresse gesprungen wird (\enquote{branch prediction} war falsch).
\begin{solutionordottedlines}[2cm]
% Solution goes here
siehe https://docs.google.com/spreadsheet/ccc?key=0AnuHoJusdbkydFB2ZjA1WDllQVl0V1BVY3A2d1M4bFE&usp=sharing

evtl. als pdf exportieren und in latex inkludieren
\end{solutionordottedlines}
%%% Next subquestion

\part
Beschreiben Sie in Ihren Worten, was ein \enquote{pipeline flush} bedeutet.
\begin{solutionordottedlines}[2cm]
% Solution goes here
Bei einem Pipeline Flush werden Ergebnisse von abgearbeiteten Teilschritten in der Pipeline verworfen, die Pipeline wird geleert und nach dem letzten gültigen Befehl wieder gefüllt.
\end{solutionordottedlines}
%%% Next subquestion

\end{parts}