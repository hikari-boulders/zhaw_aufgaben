%% Einleitung

\begin{tabularx}{\textwidth}{Xr}
  Constantin Lazari, Marco Wettstein & \today\\
\end{tabularx}

%% Fragenh
\begin{questions}
  \question
  Beispiele f�r klassische Prozessoren sind: Intel 4004, Intel 8008, Intel 8088, Intel 8086, Intel 80286, Intel 80386, Motorola 68000, Z80, MOS
6502, PowerPC 970, PDP-11, CDP1802
	\begin{parts}
		\part Geben Sie das Erscheinungsjahr sowie die intern verwendeten Wortbreite an
		\begin{solutionordottedlines}[2cm]
		\begin{center}
		\begin{tabular}{lrr}
		\multicolumn{1}{c}{\textbf{Prozessor}} & \multicolumn{1}{c}{\textbf{Jahr}} & \multicolumn{1}{c}{\textbf{Wortbreite}}\\
		\midrule
		Intel 4004 & & \\\hline
		Intel 8008 & & \\\hline
		Intel 8088 & & \\\hline
		Intel 8086 & & \\\hline
		Intel 80286 & & \\\hline
		Intel 80386 & & \\\hline
		Motorola 68000 & & \\\hline
		Z80 & & \\\hline
		MOS 6502 & & \\\hline
		PowerPC 970 & & \\\hline
		PDP-11 & & \\\hline
		CDP1802 & & \\\hline
		\end{tabular}
		\end{center}
		\end{solutionordottedlines}

		\part Wie viele verschiedene Befehle k�nnen damit dargestellt werden?
		\begin{solutionordottedlines}[2cm]		
		\end{solutionordottedlines}
		
		\part Nennen Sie pro Prozessor ein Computer-Modell bzw. Einsatzgebiet.
		\begin{solutionordottedlines}[2cm]		
		\end{solutionordottedlines}
	\end{parts}

   %%%%%%%%%%%%%%%%%%%%%%%%%%%%%%%%%%%%%%%%%%%%%%%%%%%%
	%\pagebreak
	\question
	Der \enquote{Puffer�berlauf} geh�rt zu den h�ufigsten Sicherheitsl�cken in Programmen (mit Computern mit der Von-Neumann-Architektur).
	
	  \begin{parts}
	  \part Beschreiben Sie kurz informell, warum die klassische HarvardArchitektur besser gegen diesen sch�tzt (gegen�ber der VonNeumann-Architektur).
	  \begin{solutionordottedlines}[2cm]
	  \end{solutionordottedlines}

	  \part Ist Ihre Argumentation auch bei der Super-Harvard-Architektur allgemein korrekt?
	  \begin{solutionordottedlines}[2cm]
	  \end{solutionordottedlines}

	\end{parts}

   %%%%%%%%%%%%%%%%%%%%%%%%%%%%%%%%%%%%%%%%%%%%%%%%%%%%
	%\pagebreak
	\question
	Wortbreiten

	\begin{parts}
	\part Kann ein Prozessor mit geringer Wortbreite auch Werte (bzw. Worte) berechnen, die breiter sind? Zum Beispiel ein Prozessor mit 8-BitWortbreite auch 16- oder 32-Bit-W�rter. Falls ja, wie k�nnte ein solches Verfahren aussehen?
	\begin{solutionordottedlines}[2cm]
	\end{solutionordottedlines}
	
	\end{parts}
	
	\question Architektur
	
	\begin{parts}
	\part Wieso k�nnen der Motorola 68000 und die Intel-Prozessoren 8088, 8086 und 80286 mehr als 65 KB Hauptspeicher adressieren? 
	\begin{solutionordottedlines}[2cm]
	\end{solutionordottedlines}
	
	\part Was unterscheidet den Motorola 68000 von der Architektur des Intel x86? Welche Vor- und Nachteile ergeben sich daraus?
	\begin{solutionordottedlines}[2cm]
	\end{solutionordottedlines}
	
	\end{parts}

\end{questions}

