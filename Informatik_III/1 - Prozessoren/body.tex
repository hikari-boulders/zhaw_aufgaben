%% Einleitung

\begin{tabularx}{\textwidth}{Xr}
  Constantin Lazari, Marco Wettstein & \today\\
\end{tabularx}

%% Fragenh
\begin{questions}
  \question
  Beispiele f�r klassische Prozessoren sind: Intel 4004, Intel 8008, Intel 8088, Intel 8086, Intel 80286, Intel 80386, Motorola 68000, Z80, MOS
6502, PowerPC 970, PDP-11, CDP1802
	\begin{parts}
		\part Geben Sie das Erscheinungsjahr sowie die intern verwendeten Wortbreite an. Wie viele verschiedene Befehle k�nnen damit dargestellt werden? Nennen Sie pro Prozessor ein Computer-Modell bzw. Einsatzgebiet.
		\begin{solutionordottedlines}[2cm]
		\begin{center}
		\begin{tabular}{lrrrl}
		\multicolumn{1}{c}{\textbf{Prozessor}} & \multicolumn{1}{c}{\textbf{Jahr}} & \multicolumn{1}{c}{\textbf{Wort}} & \multicolumn{1}{c}{\textbf{Befehle}} & \multicolumn{1}{c}{\textbf{Einsatzgebiet}}\\
		\midrule
		Intel 4004 & 1971 & 4 Bit & 16 & Rechenmaschinen\\\hline
		Intel 8008 & 1972 & 8 Bit & 256 & Terminals\\\hline
		Intel 8086 & 1978 & 16 Bit & 65\,536 & IBM-PC\\\hline
		Intel 8088 & 1979 & 16 Bit & 65\,536 & IBM-PC\\\hline
		Intel 80286 & 1982 & 16 Bit & 65\,536 & PCs\\\hline
		Intel 80386 & 1985 & 32 Bit & 4\,294\,967\,296 & PCs\\\hline
		Motorola 68000 & 1979 & 32 Bit& 4\,294\,967\,296 & Apple Macintosh \\\hline
		Z-80 & 1976 & 8 Bit& 256 & Arcade Spiele\\\hline
		MOS 6502 & 1975 & 8 Bit & 256 & HP-Rechner\\\hline
		PowerPC 970 & 2002 & 64 Bit & $\approx 1.8 \times 10^{19}$ & Macintosh G5\\\hline
		PDP-11 & 1970 & 16 Bit & 65\,536 & Regelungstechnik\\\hline
		CDP1802 (RCA 1802) & 1976 & 8 Bit & 256 & Raumfahrt\\\hline
		\end{tabular}
		\end{center}
Anmerkung zu Intel 4004: Die Befehlsbreite ist 8-Bit, es stehen 46 Befehle zur Verf�gung.
		\end{solutionordottedlines}
	\end{parts}

   %%%%%%%%%%%%%%%%%%%%%%%%%%%%%%%%%%%%%%%%%%%%%%%%%%%%
	%\pagebreak
	\question
	Der \enquote{Puffer�berlauf} geh�rt zu den h�ufigsten Sicherheitsl�cken in Programmen (mit Computern mit der Von-Neumann-Architektur).
	
	  \begin{parts}
	  \part Beschreiben Sie kurz informell, warum die klassische Harvard Architektur besser gegen diesen sch�tzt (gegen�ber der VonNeumann-Architektur).
	  \begin{solutionordottedlines}[2cm]
	  Bei der klassischen Harvard Architektur sind Programmcode (Programmdaten) und Nicht-Programmdaten in physisch getrennten Speichern abgelegt. Bei der Von-Neumann-Architektur sind Code und Daten aber im gleichen Speicher abgelegt. Werden dabei Daten von externen Quellen (Datentr�ger, Tastatur, oder �hnliches) in den Arbeitspeicher geladen, muss das Programm einen bestimmten Speicherbereich (Puffer) f�r diese Daten reservieren. Sind die geladenen Daten gr�sser als dieser Bereich, ist es m�glich, dass das Programm Teile des Programmcodes im Speicher �berschreibt. So ist es m�glich, �ber eingelesene Daten, den Ablauf des Programmes zu manipulieren. Bei der Harvard-Architektur ist dies nicht m�glich, weil eingelesene Daten in einen anderen Speicher geladen werden, als der Programmcode.
	  \end{solutionordottedlines}

	  \pagebreak
	  \part Ist Ihre Argumentation auch bei der Super-Harvard-Architektur allgemein korrekt?
	  \begin{solutionordottedlines}[2cm]
	  Nein, denn bei der Super-Harvard-Architektur sind Programmdaten und Nicht-Programmdaten wieder im selben Speicher untergebracht. Es existiert lediglich eine Trennung von Program-Bus (f�r Programmdaten) und Daten-Bus (f�r Nicht-Programmdaten).
	  \end{solutionordottedlines}

	\end{parts}

   %%%%%%%%%%%%%%%%%%%%%%%%%%%%%%%%%%%%%%%%%%%%%%%%%%%%
	%\pagebreak
	\question
	Wortbreiten

	\begin{parts}
	\part Kann ein Prozessor mit geringer Wortbreite auch Werte (bzw. Worte) berechnen, die breiter sind? Zum Beispiel ein Prozessor mit 8-Bit Wortbreite auch 16- oder 32-Bit-W�rter. Falls ja, wie k�nnte ein solches Verfahren aussehen?
	\begin{solutionordottedlines}[2cm]
	Offensichtlich ist dies m�glich, sonst k�nnte nicht mit sehr sehr grossen Werten gerechnet werden k�nnen.\\
	M�gliches Verfahren: Man h�ngt einfach mehrere Speicherregister hintereinander. Beispielsweise besteht dann ein 16-Bit Wort in Wirklichkeit aus 2 8-Bit-W�rtern. Wenn das Programm weiss, womit es es jeweils zu tun hat und ein entsprechendes Handling implementiert ist, besteht theoretisch kein Problem.
	\end{solutionordottedlines}
	
	\end{parts}
	
	\question Architektur
	
	\begin{parts}
	\part Wieso k�nnen der Motorola 68000 und die Intel-Prozessoren 8088, 8086 und 80286 mehr als 65 KB Hauptspeicher adressieren? 
	\begin{solutionordottedlines}[2cm]
	Der Speicher wird in jeweils 64kb grosse Segmente unterteilt, welche intern mittels 16bit grossen Adressen addressiert werden kann. Einer 16bit Adresse wird dabei eine wiederum 16bit grosse Segmentadresse vorangestellt. Die Segmentadresse im Segmentregister bestimmt dabei das zu w�hlende Speichersegment, die Adresse im Adressregister die konkrete Speicherzelle.
	\end{solutionordottedlines}
	
	\part Was unterscheidet den Motorola 68000 von der Architektur des Intel x86? Welche Vor- und Nachteile ergeben sich daraus?
	\begin{solutionordottedlines}[2cm]
	Der Motorola 68k enth�lt einen grunds�tzlich anderen Befehlssatz als die Intel x86er Prozessoren. Operationen und Operanden k�nnen frei kombiniert werden. Es gibt also keine eigenen Befehle f�r \enquote{diese Operation} mit \enquote{jenem Operand}. Im Ergebnis ist der Motorola 68k einfacher zu programmieren.
	\end{solutionordottedlines}
	\end{parts}

\end{questions}

